\documentclass[english]{revtex4-1}
\usepackage[T1]{fontenc}
\usepackage[latin9]{inputenc}
\setcounter{secnumdepth}{3}
\usepackage{float}
\usepackage{graphicx}
\usepackage{babel}
\usepackage{comment}

%\usepackage[
%backend=biber,
%style=alphabetic,
%sorting=ynt
%]{biblatex}

\usepackage{biblatex}

\addbibresource{references.bib}



\begin{document}
\title{Prelim outline draft.}


\section{Introduce gaps in knowledge.}

\begin{enumerate}
    
    \item Introduce topological materials.

    \begin{itemize}
    
        \item Introduce topological insulators and their peculiar features.
    
    \end{itemize}
    
    \item Introduce Weyl semimetals and topological Dirac semimetals.
    
    \begin{itemize}
    
        \item What differentiates these materials from topological insulators?
    
        \item What differentiates Weyl semimetals from topological Dirac semimetals?
    
    \end{itemize}
    
    \item We focus on a particular topological Dirac semimetal, Na$_3$Bi. 
    
    \begin{itemize}
    
        \item Discuss its properties like strong spin-orbit coupling, and time-reversal and inversion symmetries.
        
       \item Natural topological Dirac semimetals like Na$_3$Bi have surface states that do not follow the bulk-boundary correspondence that governs Fermi arc states in Weyl semimetals. 
        
    \end{itemize}    

    \item Gaps in knowledge: existence of Fermi arcs in Na$_3$Bi.
    
    \begin{itemize}
    
        \item Some papers discussed whether Fermi arcs in Dirac semimetals, like Na$_3$Bi, are thought to exist on surfaces whose surface normals are not along the z-axis (along surfaces with normals along the z axis, the Dirac points and their associated Fermi arc states project onto a single Dirac point). \cite{kargarian_are_2016}

        \item By studying the symmetries of certain planes  of Na$_{3}$Bi under K-theory analysis as well as effective field theory analysis, the authors of \cite{kargarian_are_2016} indicate certain surfaces like the (110) surface of Na$_{3}$Bi possess Fermi arc states. (Cite PNAS paper and related literature).

        \item A more detailed analysis involving density functional theory (DFT) with spin-orbit coupling has not been performed to support such claims.

    \end{itemize}
   
     \item Gaps in knowledge: band gap oscillations in  Na$_3$Bi.
   
    \begin{itemize}
        
        \item Slabs of Na$_{3}$Bi exhibit band gap oscillations that decay with slab thickness. These oscillations are observed along the direction of confinement. \cite{xiao_anisotropic_2015}
        
        \item However, this observation was based on a tight-binding model.\cite{xiao_anisotropic_2015} A more rigorous analysis involving DFT that can either support or deny such claims would be welcomed.
        
        \item Do the band gap oscillations exist for k-space paths along other Na$_{3}$Bi surfaces?
        
    \end{itemize}



\end{enumerate}


\section{Demonstration of research done...}

\section{Proposal.}

\begin{enumerate}
    
    \item Can we find topological states in a heterostructure of KTaO$_3$/Na$_3$Bi?
    
    \item  How do surface states of a type-I Weyl semimetal such as TaAs interact with those of Na$_3$Bi in a heterostructure of the two?  
    
    \item  How do surface states of a type-II Weyl semimetal such as WTe$_2$ interact with those of Na$_3$Bi in a heterostructure of the two?  
    
    \item In heterostructures comprised of a ferromagnetic or antiferromangetic material and Na$_3$Bi, are the spin orbit torques affected by the presence of both gapless bulk and surface states?
    
    \item Is the out-of-plane spin accumulation, which is usually affected by the bulk states, have contributions from Na$_3$Bi's  Fermi arc states?
    
\end{enumerate}


\medskip

\printbibliography

\end{document}